The Communication\+Module is a library aimed at 8bit-\/avr microcontrollers. The intent is to offer software support for several 802.\+15.\+4 based network chips like \href{https://www.microchip.com/wwwproducts/en/MRF24J40MA}{\tt M\+R\+F24\+J40\+MA} or the \href{https://www.digi.com/products/xbee-rf-solutions/2-4-ghz-modules/xbee-802-15-4#productsupport}{\tt X\+Bee}. To allow interaction with these chips a drivers for the respective peripheral interface (U\+S\+A\+RT, S\+PI) are included.

\subsection*{Dependencies}

The library comes with very few but essential dependencies. To use the precompiled library you\textquotesingle{}ll need
\begin{DoxyEnumerate}
\item avr-\/gcc
\item \href{https://github.com/ThrowTheSwitch/CException}{\tt C\+Exception}
\end{DoxyEnumerate}

To build our library you\textquotesingle{}ll have to install \href{https://bazel.build}{\tt Bazel} and create the C\+R\+O\+S\+S\+T\+O\+O\+LS file to enable usage of avr-\/gcc. For more information on this see this \href{http://confluence.es.uni-due.de:8090/pages/viewpage.action?pageId=23953429}{\tt confluence article} .

\subsection*{How to use the library}

The library follows a strict separation of interfaces and implementation. Several different implementations of each interface may be in use at the same time (however keep in mind not to use the same physical ressource more than once). All implementation is hidden behind abstract data types. To start using a module you have to create the corresponding structs. To give users as much control over their memory usage as possible, every implementation offers two functions


\begin{DoxyEnumerate}
\item size\+\_\+t Interface\+Name\+Implementation\+Name\+\_\+get\+A\+D\+T\+Size(void);
\item Interface\+Name Interface\+Name\+\_\+create(\+Interface\+Name ptr\+\_\+to\+\_\+memory, Optional\+Config\+Parameters parameters);
\end{DoxyEnumerate}

The create function usually also initializes the implementation, so that after calling it you can start using the implementation. For details about functions offered by the interfaces see their doxygen documentation or take a look at their header files.

\subsubsection*{Exceptions}

Instead of passing and handling error codes in long if-\/else statements, we use the C\+Exception library. However currently it is only partially used.

\subsection*{Known Issues}


\begin{DoxyItemize}
\item non blocking functions are in development
\item enabling promiscuous mode seems to prevent back to back reception of packages 
\end{DoxyItemize}